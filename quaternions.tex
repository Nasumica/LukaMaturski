\subsection{Kvaternioni}

\def\uv{{u}}
\def\vp{{\vec v}}
\def\norm#1{{\vert#1\vert}}
\def\con#1{{\bar#1}}

Poput skupa kompleksnih brojeva $\Cset$, koji predstav{\lj}a objekte u 2D prostoru,
skup {\sl kvater\-niona\/}~$\Hset$, predstav{\lj}a objekte u 3D prostoru. Prvi ih je opisao 
1843. godine irski ma\-te\-ma\-ti\-{\cv}ar
Hamilton (William Rowan Hamilton), te {\nj}emu u {\cv}ast i oznaka skupa $\Hset$.
U~informatici su neizbe{\zv}ni deo svega {\sv}to se de{\sv}ava u 3D.

% \subsubsection{Definicija}
\medskip

Kvaternion $q\in\Hset$ mo{\zv}e biti predstav{\lj}en kao zbir
\begin{equation}
    q=s+v
\end{equation}
koji se sastoji od skalarnog dela $s\in{\mathbb R}$ i vektorskog dela $v\in{\mathbb R}^3$, gde je
\begin{equation}
    v=(x,y,z)=xi+yj+zk
\end{equation}
3D vektor, 
a gde su $i$, $j$ i $k$ jedini{\cv}ni vektori po $x$, $y$ i $z$ osi,
za koje va{\zv}i
\begin{gather}\label{eq:qunits}
    i^2=j^2=k^2=ijk=-1,\quad
    ij=k,\quad jk=i,\quad ki=j. 
\end{gather}

\danger U skupu kvaterniona $\Hset$ za operaciju mno{\zv}e{\nj}a, uop{\sv}teno, ne va{\zv}i {\sl zakon komutacije}:
$ji=-ij=-k$, $kj=-jk=-i$, $ik=-ki=-j$.
Ovo je logi{\cv}no kad se setimo da i kod
{\sl Rubikove kocke\/} naj{\cv}e{\sv}{\cc}e nije svejedno kojim redosledom okre{\cc}emo stranice.
Va{\zv}i {\sl asocijativnost}: $(p\cdot q)\cdot r=p\cdot(q\cdot r)$.

% \subsubsection{Operacije i funkcije}
\medskip

Da bi $q=s+v$ bio {\sl pravi\/} kvaternion, mora biti $v\ne0$, ina{\cv}e je $q$ obi{\cv}an realan broj, 
kada se prime{\nj}uju operacije i funkcije iz skupa $\Rset$.
Ako odredimo apsolutnu vrednost kvaterniona
$$
\lambda = \norm v = \sqrt{\mathstrut x^2+y^2+z^2},\qquad
\rho = \norm q = \sqrt{\mathstrut s^2+\lambda^2}
$$
koja se zove {\sl norma\/}, odredimo
{\sl jedini{\cv}ni vektor\/} ({\sl unit\/}) vektorskog dela kvaterniona
$$
\uv = \frac v\lambda, 
$$
koji se zove {\sl versor}
i gde je po definiciji%
\footnote{U skupu $\Hset$, $\sqrt{-1}$ ima beskona{\cv}no re{\sv}e{\nj}a:
svaki kvaternion koji se nalazi na jedini{\cv}noj sferi je re{\sv}e{\nj}e
($s=0\land x^2+y^2+z^2=1$), odnosno, svaki versor.} 
$\norm\uv=1$ i  $\uv^2=-1$,
kao i ugao orijentacije
$$
\varphi = \arccos\left( \frac s\rho \right),
$$
mo{\zv}emo dobiti polarni zapis kvaterniona
\begin{equation}
    q
=\rho\,(\cos\varphi + \uv\sin\varphi)
=\rho\,\e^{\uv\varphi}.
\end{equation}
Iz svega ovoga se mo{\zv}e dobiti
\begin{equation}
    \okvir{\exp(q) = \e^s \left( \cos\lambda + \uv\sin\lambda \right)}
\end{equation}
\hbox to \textwidth{\hss i\hss}
\begin{equation}
    \okvir{\ln(q)  = \ln\rho + \uv\varphi}\rlap{.}
\end{equation}

\medskip

\danger
Ostale operacije i funkcije nisu tema ovog rada, ali sabira{\nj}e i oduzima{\nj}e
je uobi\-{\cv}a\-jeno, za mno{\zv}e{\nj}e treba obratiti pa{\zv}{\nj}u na formulu \eqref{eq:qunits} i
komutativnost, a recipro{\cv}na vrednost je
$q^{-1}=\con q/\rho^2$, gde je $\con q=s-v$, {\sl konjugovana\/} vrednost. 
% Naj{\cv}e{\sv}{\cc}e se de{\lj}e{\nj}e ne pi{\sv}e $p/q$, ve{\cc} $pq^{-1}$.
Tri\-gono\-metrijske i hiperboli{\cv}ne funkcije se mogu izraziti pomo{\cc}u eksponencijalne,
a {\nj}ihove inverzne pomo{\cc}u logaritamske funkcije.\par

