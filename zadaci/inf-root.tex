\subsubsection{Beskona{\cv}ni koren}

\zadatak
Odredi vrednost
\begin{align*}
    x&=\e\sqrt[2]{\e\sqrt[3]{\e\sqrt[4]{\e\sqrt[5]{\cdots}}}}
\intertext{
\resenje Izvr{\sv}imo zamenu $t=\ln x$, 
a kako je $\ln\e=1$ i koriste{\cc}i jednakost za logaritam proizvoda \eqref{eq:lnmul} 
i jednakost za loga\-ri\-tam korena \eqref{eq:lnroot}, mo{\zv}emo pisati}
%&=\ln\left(\e\sqrt[2]{\e\sqrt[3]{\e\sqrt[4]{\e\sqrt[5]{\cdots}}}}\right)\\
t&=\ln\left(\e\sqrt[2]{\e\sqrt[3]{\e\sqrt[4]{\e\sqrt[5]{\cdots}}}}\right)\\
&=1+\frac12\ln\left(\e\sqrt[3]{\e\sqrt[4]{\e\sqrt[5]{\cdots}}} \right)\\
&=1+\frac12\left(1+\frac13\ln\left(\e\sqrt[4]{\e\sqrt[5]{\cdots}} \right)\right)\\
&=1+\frac12\left(1+\frac13\left(1+\frac14\ln\left(\e\sqrt[5]{\cdots} \right) \right)\right)\\
&=1+\frac12\left(1+\frac13\left(1+\frac14\left(1+\frac15\ln(\cdots)\right) \right) \right)\\
&=1+\frac12+\frac12\mul\frac13+\frac12\mul\frac13\mul\frac14+\frac12\mul\frac13\mul\frac14\mul\frac15+\cdots\\
&=\rf1!+\rf2!+\rf3!+\rf4!+\rf5!+\cdots
\intertext{Ako pogledamo formulu \eqref{eq:e} na strani \pageref{eq:e}, mo{\zv}emo videti da je ovaj zbir jednak}
t&=\e-1,
\intertext{jer iz sume za izra{\cv}unava{\nj}e $\e$ nedostaje {\sl nulti\/} {\cv}lan $1/0!=1$.
Odavde je}
x&=\ram{\e^{\e-1}}\approx 5\.575.
\end{align*}
