\subsubsection{Izvod 1}

\zadatak Odredi izvod funkcije $f(\alpha)=\log_3(\cos\alpha)$.
\resenje Kako je izvod $\cos\alpha$ jednak $-\sin\alpha$ i iz jednakosti 
\eqref{eq:izvod} za izvod logaritma funkcije, sledi da je
$$
f'(\alpha)
=\frac{-\sin\alpha}{\cos\alpha\ln3}
=\ram{-\frac{\tan\alpha}{\ln3}}.
$$

% \zadatak  Odredi izvod funkcije
% \begin{align*}
%     f(x) &= \ln\left(\frac{x^5 (1-\sin^2 x)}{3x+1}\right).
% \intertext{\resenje Zamenimo $1-\sin^2x=\cos^2 x$ i razlo{\zv}imo funkciju}
% f(x) &= \ln x^5 + \ln\cos^2 x - \ln(3x+1)\\
%  &= 5\ln x + 2\ln\cos x - \ln(3x+1).
% \intertext{Iz jednakosti \eqref{eq:izvod} za izvod logaritma funkcije, sledi da je}
%     f'(x) &= \frac5x + 2\cdot\frac{-\sin x}{\cos x} - \frac3{3x+1}\\
%     &= \ram{\frac{12x + 5}{3x^2+x} - 2\tan x }.
% \end{align*}

\subsubsection{Izvod 2}

\zadatak Odredi izvod funkcije $\displaystyle {f(x)=\ln\left(\frac{x^2}{x+1}\right)}$.

\resenje Ako logaritam koli{\cv}nika {\sl na\"\i vno\/} predstavimo kao razliku logaritama
$$
f(x) = 2\ln x - \ln (x+1),
$$
dobili bi smo 
$$
f'(x) = \frac2x + \frac1{x+1} = {\color{red}\frac{3x+2}{x(x+1)}},
$$
{\sv}to je pogre{\sv}no. Da bismo izra{\cv}unali ispravno, moramo prvo da za argument
logaritma $x^2/(x+1)$ odredimo izvod, koji je $x(x+2)/(x+1)^2$, odakle sledi
$$
f'(x)=\frac{x(x+2)/(x+1)^2}{x^2/(x+1)}=\ram{\frac{x+2}{x(x+1)}}.
$$
