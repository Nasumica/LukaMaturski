\subsubsection{{\Cv}etiri {\cv}etvorke}

\def\4{{\color{red}\bf4}}\index{4@\4}

\zadatak
Dokazati da svaki prirodan broj $n\in\Nset$, mo{\zv}e biti predstav{\lj}en sa \4 broja \4,
pomo{\cc}u logaritamske funkcije i kvadratnog korena\index{koren $(\sqrt x)$}
$$
n=\log_{\sqrt\4/\4}\left(\log_\4 \underbrace{\sqrt{\sqrt{\cdots\sqrt\4}}}_{\text{$n$ korena}}\right).
$$

\resenje
Kako je
$$
\frac{\sqrt \4}{\4}=\frac12
\qquad\text{i}\qquad
\underbrace{\sqrt{\sqrt{\cdots\sqrt \4}}}_{\text{$n$ korena}}=\4^{(1/2)^n},
$$
izraz mo{\zv}e biti upro{\sv}{\cc}en
\begin{align*}
\noalign{\vskip-9pt}
\log_{\sqrt\4/\4}\left(\log_\4 \underbrace{\sqrt{\sqrt{\cdots\sqrt\4}}}_{\text{$n$ korena}} \right)
&=\log_{1/2}\left(\log_\4 \4^{(1/2)^n}\right),\\
\intertext{gde iz jednakosti za logaritam stepena osnove \eqref{eq:powb}, sledi}
&=\log_{1/2}(1/2)^n\\
&=\ram{n}.
\end{align*}

\def\2{{\it2}}\index{2}\QEDidx
\def\dlog{\mathop{\it\ell\mskip -0.5\thinmuskip og}\nolimits_\2}
\dodatak\begingroup\it
Davno je u jednom {\cv}asopisu postav{\lj}en sli{\cv}an zadatak: 
da se sa {\sv}to ma{\nj}e istih brojeva,
koriste{\cc}i bilo koju matemati{\cv}ku funkciju, predstavi 
svaki prirodan broj $n$.
Re{\sv}io ga je nobelovac Pol \idx{Dirak} (Paul Dirac) 
sa~3~broja~\2, {\cv}ije originalno
re{\sv}e{\nj}e izgleda
$$
-\dlog\dlog\sqrt{\cdots n\cdots\sqrt\2\index{log2@$\log_2$}}
=-\dlog\dlog\2^{\2^{-n}}
=-\dlog\2^{-n}=n.
\eqno{\QED}$$
\endgroup
