\subsubsection{ln 3}\label{sssec:ln3}
 
\zadatak
Pomo{\cc}u postupka \eqref{eq:alg} sa 
strane~\pageref{eq:alg},
izra{\cv}unati {\sl pe{\sv}ke\/} pribli{\zv}nu vrednost $\ln 3$, 
u~5 koraka. Za upore{\dj}iva{\nj}e, ta{\cv}na vrednost je
$$
\ln3=1\.
0986122886\,
6810969139\,
5245236922\,
5257046475\,\ldots
%1639157/1492025=1.098612288668
$$

\def\step#1{\par\smallskip\indent\leavevmode
  Korak~{\it#1}.\kern2em\relax}

\resenje
Za $x=3$, izraz $r=(x-1)/(x+1)=1/2$. U nultom koraku postav{\lj}amo po{\cv}etne vrednosti:
\step0 $k=1,\quad p=2r=1,\quad q=r^2=1/4,\quad a=p=1,\quad y=a=1$.

\smallskip
\noindent Slede koraci iteracije --- pove{\cc}amo $k$ za 2, pomno{\zv}imo $p$ sa $q$,
{\cv}lan sume $a$ postaje $p/k$, koga dodajemo u rezultat $y$:

\step1 $k=3,\quad p=1/4,\quad a=1/12,\quad y=13/12$;
\step2 $k=5,\quad p=1/16,\quad a=1/80,\quad y=263/240$;
\step3 $k=7,\quad p=1/64,\quad a=1/448,\quad y=7379/6720$;
\step4 $k=9,\quad p=1/256,\quad a=1/2304,\quad y=88583/80640$;
\step5 $k=11,\quad p=1/1024,\quad a=1/11264,\quad y=3897967/3548160$.

\medskip
\noindent Rezultat je
$$
\ln3\approx{3897967\over3548160}\approx \ram{1\.0986},
$$

\smallskip\noindent
{\sv}to je prili{\cv}no ta{\cv}no za samo 5 koraka, jer je apsolutna gre{\sv}ka oko $2\.4\puta10^{-5}$.

\iffalse
Da smo ra{\cv}unali u 20 koraka, 
dobili bi smo $\ln3\approx\frac{636083906982236368109838473}{578988523561291667944243200}$, 
gre{\sv}ka bi bila oko $7\puta10^{-15}$, {\sv}to je preciznost s kojom je 1617.\ godine Brigs
izra{\cv}unao svoje prve logaritamske tablice.
\fi

\dodatak
Ako ve{\cc} imamo precizno izra{\cv}unatu konstantu $\ln2$, onda je bo{\lj}e ra{\cv}unati $\ln3$ kao $\ln(3/2)+\ln2$,
jer {\cc}e u postupku, umesto $r=1/2$, biti $r=(3/2-1)/(3/2+1)=1/5$, 
odnosno, umesto $q=1/4$, bi{\cc}e $q=1/25$,
{\sv}to dovodi do mnogo br{\zv}eg izra{\cv}unava{\nj}a. U istom broju koraka bi smo dobili
$$
\ln\frac32\approx
\ff2/5, \ff152/375, \ff19006/46875, \ff665216/1640625, \ff49891214/123046875, 
\ff13720083976/33837890625,
$$
gde posled{\nj}i razlomak ima gre{\sv}ku od oko $1\.3\puta10^{-10}$.
Kada mu dodamo $\ln2$ sa strane~\pageref{ln2} dobi{\cc}emo
$
\ln3\approx1.0986122885\ldots
$
\iffalse
Ovakav postupak je najbr{\zv}i ako ra{\cv}unamo $\ln x=\ln(x/2^n)+n\ln2$,
gde biramo $n$ takvo da $|x/2^n|$ bude {\sv}to bil{\zv}e 1, odnosno, da $|q|$ bude najma{\nj}e mogu{\cc}e.
Na primer, za $\ln5$ treba izabrati $n=2$, odnosno, $\ln5=\ln(5/2^2)+2\ln2$, dok je
$\ln7=\ln(7/2^3)+3\ln2$.
\fi