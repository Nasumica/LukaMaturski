\subsubsection{Analogni kvadratni koren}\label{sssec:sibersqrt}

\zadatak
Objasni na{\cv}in za odre{\dj}iva{\nj}e vrednosti $\sqrt x$ logaritmarom.

\resenje Kvadratni koren se mo{\zv}e direktno {\cv}itati sa logaritmara ako {\sl u glavi\/} izvr{\sv}imo 
de{\lj}enje sa 2, i po potrebi, sabira{\nj}e sa $0\.5$, 
{\sv}to je vrlo jednostavno, jer se radi o brojevima izme{\dj}u 0 i 1 sa najvi{\sv}e 3 decimale. 
Kako je
$$
\sqrt x=10^{\frac12\log x},
$$
potrebno je pro{\cv}itati vrednost $\log x$, a onda za dvostruko ma{\nj}u vred\-nost od {\nj}e, pro\-{\cv}i\-ta\-ti vrednost $10^y$. Na primer,
za izra{\cv}unava{\nj}e $\sqrt{5\.3}$,
{\cv}itamo da je $\log 5\.3\approx0\.724$~($\color{red}\downarrow$), a onda {\cv}itamo vrednost $10^y$ za 
$y=0\.724/2=0\.362$~($\color{red}\uparrow$) i
dobijamo $\sqrt{5\.3}\approx 2\.3$.
$$
\def\hair{\nit{0.36213793480039452281649614581363}\nit{0.72427586960078904563299229162726}\nit{0.86213793480039452281649614581363}}
\def\extra{\ruler{0.36213793480039452281649614581363}{$\uparrow$}\ruler{0.72427586960078904563299229162726}{$\downarrow$}\ruler{0.86213793480039452281649614581363}{$\color{magenta}\uparrow$}}
\logaritmar0
$$
Za $\sqrt{53}$ treba u $y$ dodati jo{\sv} $0\.5$ tako da je $y=0\.362+0\.5=0\.862$~($\color{magenta}\uparrow$), odakle je $\sqrt{53}\approx7\.28$.
%(Dodava{\nj}e $0\.5$ u eksponent predstav{\lj}a mno{\zv}e{\nj}e celog izraza sa $\sqrt{10}$.)
Naravno, $\sqrt{530}$ se ra{\cv}una kao $10\sqrt{5\.3}$, ili $\sqrt{0\.53}=\frac{1}{10}\sqrt{53}$.
