\subsubsection{ln($-$\textit{z})}

\zadatak U skupu kompleksnih brojeva $\mathbb C$, ako znamo vrednost $\ln z$,
koliko je $\ln(-z)$?

\resenje Kako je u kompleksnoj ravni $-z$ jednako $z$ zarotirano 
oko koordi\-natnog po{\cv}etka
za ugao od $180^\circ=\pi$, dobijamo
$$
\ln (-z) = \ram{\ln z +i\pi}.
$$ 
Ako proverimo, iz Ojlerove jedna{\cv}ine \eqref{eq:zeuler}, dobijamo
$$
\e^{\ln(-z)} = \e^{\ln z+i\pi} = \e^{\ln z}\cdot \e^{i\pi}=z \cdot (-1)=-z.
$$

\dodatak Hmmm \dots, trik sa rotacijom nije ba{\sv} potpuno ta{\cv}an: dobili bismo $-z$ i za ugao $-\pi$,
pa bi bilo $\ln(-z)=\ln z-i\pi$, {\sv}to je tako{\dj}e ta{\cv}no;
u stvari, ta{\cv}no je za bilo koji ugao $\pi+2k\pi$ gde je $k\in{\mathbb Z}$ ceo broj. Odavde bi sledilo da je
$$
\ln(-z) = \ln z + i(\pi+2k\pi),\ k\in{\mathbb Z}.
$$

I sama formula \eqref{eq:cln}, $\ln z=\ln\rho + i\theta$, predstav{\lj}a samo {\sl glavnu granu\/}
kompleksnog logaritma, koji je nekakva vrsta 4D spirale. Potpuna formula bi bila
\begin{equation}
    \okvir{\ln z=\ln\rho + i(\theta + 2k\pi),\ k\in{\mathbb Z}} .
\end{equation}