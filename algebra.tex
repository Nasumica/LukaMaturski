
% \newcount\brk \brk=\value{section}
% \renewcommand\thesection{\Alph{section}}

\section{Logaritam u algebri}

\subsection{Kvaternioni}

\def\uv{{\hat u}}
\def\vp{{\vec v}}
\def\norm#1{{\Vert#1\Vert}}

Poput skupa kompleksnih brojeva $\Cset$, koji predstav{\lj}a ta{\cv}ke u 2D prostoru,
skup~$\Hset$\break {\sl kvaterniona\/} predstav{\lj}a ta{\cv}ke u 3D prostoru. Prvi ih je opisao 
1843. godine irski ma\-te\-ma\-ti\-{\cv}ar
Hamilton (William Rowan Hamilton), te {\nj}emu u {\cv}ast i oznaka skupa $\Hset$.

\subsubsection{Definicija}

Kvaternion $q\in\Hset$ mo{\zv}e biti predstav{\lj}en kao zbir
\begin{equation}
    q=s+v
\end{equation}
koji se sastoji od skalarnog dela $s\in{\mathbb R}$ i vektorskog dela $v\in{\mathbb R}^3$, gde je
\begin{equation}
    v=xi+yj+zk
\end{equation}
3D vektor, a gde su $i$, $j$ i $k$ jedini{\cv}ni vektori po $x$, $y$ i $z$ osi,
za koje va{\zv}i
\begin{gather}    
    i^2=j^2=k^2=ijk=-1,\quad
    ij=k,\quad jk=i,\quad ki=j. 
\end{gather}

\danger U skupu kvaterniona $\Hset$ za operaciju mno{\zv}e{\nj}a, uop{\sv}teno, ne va{\zv}i {\sl zakon komutacije}:
$ji=-ij=-k$, $kj=-jk=-i$, $ik=-ki=-j$.
Ovo je logi{\cv}no ako se setimo da i kod
{\sl Rubikove kocke\/} nije svejedno kojim redosledom okre{\cc}emo stranice.
Va{\zv}i {\sl asocijativnost}: $(i\cdot j)\cdot k=i\cdot(j\cdot k)$.

\subsubsection{Operacije i funkcije}

Analogno kompleksnim brojevima, mo{\zv}e se definisati apsolutna vrednost
$$
\rho = \norm q = \sqrt{s^2+x^2+y^2+z^2}
$$
(koja se u ovom slu{\cv}aju zove {\sl norma\/}),
ugao rotacije
$$
\varphi = \arccos\left( \frac s\rho \right)
$$
i {\sl jedini{\cv}ni\/} ({\sl unit\/}) vektor vektorskg dela kvaterniona
$$
\uv = \frac v{\norm v},
$$
odakle sledi polarni zapis kvaterniona
\begin{equation}
    q
=\rho\,(\cos\varphi + \uv\sin\varphi)
=\rho\,\e^{\uv\varphi}.
\end{equation}
Iz svega ovoga mo{\zv}e se dobiti
\begin{equation}
    \okvir{\exp(q) = \e^s \left( \cos\norm v + \uv\sin\norm v \right)}
\end{equation}
i
\begin{equation}
    \okvir{\ln(q)  = \ln\rho + \uv\varphi}.
\end{equation}

\medskip

Ostale operacije i funkcije nisu tema ovog rada.


\newpage

\subsection{Matrice}

\newpage
